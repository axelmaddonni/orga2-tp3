\subsubsection*{Estructuras del juego}

Analicemos ahora las estructuras que utilizamos para llevar a cabo el juego.


\begin{lstlisting}
typedef struct pirata_t
{
    uint index;
    struct jugador_t *jugador;
    
    tipo_t tipo;
    
    uint id_pirata;
    uint posicion[2];
    
    uint cr3;

    uint estado_reloj;

} pirata_t;
\end{lstlisting}

\texttt{index} es el indice del arreglo de jugadores del jugador al que corresponde este pirata y \texttt{jugador} es un puntero al jugador correspondiente.

\texttt{tipo} es MINERO o EXPLORADOR e indica de que tipo es el pirata.

\texttt{id_pirata} es la id unica del pirata (y de la tarea), mientras que \texttt{posicion} es su ubicación dentro del juego.

\texttt{cr3} es el cr3 de la tarea. Lo vamos a usar cuando algun pirata explorador descubra nuevos lugares, dado que necesitamos mapearle esas nuevas páginas a todos los jugadores del juego.

\texttt{estado_reloj} nos va a permitir saber cual es el estado del reloj de ese jugador para poder ir girandolo (ver screen.h).


\begin{lstlisting}
typedef struct jugador_t
{
    cual_t jug;
    
    pirata_t piratas[MAX_CANT_PIRATAS_VIVOS];
    uchar vivos[MAX_CANT_PIRATAS_VIVOS];
    
    uchar color;

    struct {
      minero_obj ms[10]; 
      uint proximo_a_ejecutar;
      uint proximo_libre;
      
    } mineros_pendientes;
 
    uint monedas;
  
    uchar posiciones_exploradas[MAPA_ALTO][MAPA_ANCHO];
    int puerto[2];   
} jugador_t;
\end{lstlisting}

\texttt{jug} puede valer A (0) o B (1), y nos indica que jugador es este.

\texttt{piratas} y \texttt{vivos} son dos arreglos que nos indican (para cada uno de los 8 piratas que podemos lanzar en total), la información del pirata (como vimos antes) y si ese pirata esta vivo o no. Si no está vivo vamos a poder lanzar nuevos piratas y reutilizar ese slot.

\texttt{color} es el color del cual se tiene que imprimir el fondo de la pantalla cuando un pirata de ese jugador se mueva.

\texttt{mineros_pendientes} es una estructura que contiene un arreglo de mineros, y nos permite saber cuantos quedan por ejecutar.

\texttt{monedas} son la cantidad de monedas que tiene el jugador en un momento dado.

\texttt{posiciones_exploradas} es una matriz que nos permite saber que posiciones exploró y cuales no el jugador.

\texttt{puerto} son las coordenadas del puerto del jugador.


