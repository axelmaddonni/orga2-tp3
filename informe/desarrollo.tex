\par La explicación de la implementación del sistema está dividida según los ejercicios planteados por el enunciado. Al final, se encuentra un apéndice con una breve explicación sobre las estructuras del juego creadas para almacenar la información del juego.
\par A continuación, cómo está dividido el desarrollo y qué se encuentra en cada ejercicio:
\begin{itemize}
\item Ejercicio 1: Inicialización de la GDT, Pasaje a Modo Protegido e Inicialización de la Pantalla. 
\item Ejercicio 2: Inicialización de la IDT.
\item Ejercicio 3: Inicialización de directorio y tablas de páginas de kernel y Activación de Paginación. 
\item Ejercicio 4: Inicialización de la MMU, Mapeo y Desmapeo de Páginas, Inicialización de directorios y tablas para tareas Pirata.
\item Ejercicio 5: Interrupción de Reloj, Interrupción de Teclado e Interrupción de syscalls 0x46.
\item Ejercicio 6: Inicialización de las TSS, Rutinas de atención de servicios para MOVER, CAVAR y calcular POSICION de piratas, ejecución de tareas.
\item Ejercicio 7: Inicialización del Scheduler y sus funciones, Modo Debugger.
\item Apéndice: Descripción de los struct_Pirata y struct_Jugador para almacenar información sobre el juego, y funciones auxiliares.
\end{itemize}