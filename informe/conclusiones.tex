Para concluir el informe de este trabajo práctico, podemos decir que durante su desarrollo aprendimos y entendimos como funcionan en la realidad muchos mecanismos básicos de la computadora, desde como se maneja la memoria, hasta como conmutar tareas.

Sin embargo también entendimos que los sistemas operativos modernos son varios órdenes de maginitud mas complejos que el que nos tocó hacer. Sin embargo las ideas son las mismas, y creemos que si nos embarcaramos a leer el código de cualquier sistema operativo moderno, entenderíamos mucho más que si nunca hubieramos hecho este TP.

Algo que nos llamó la atención particularmente fue como funciona el sistema de memoria virtual, que nosotros no conocíamos y permite entender muchas cosas de las que suceden en nuestras computadoras (particion swap, por ejemplo).

Para terminar, que nos gustaría decir es que nos parece importante notar es como algunas cosas que vimos en la carrera se pueden aplicar a este area: la demostración de correctitud de programas es una idea que se podría aplicar tranquilamente en las partes mas centrales de los sistemas operativos. Esto permitiría que los sistemas operativos modernos sean mucho mas confiables y además eliminaría lugares potenciales donde buscar bugs, entonces sería más fácil
encontrarlos.



