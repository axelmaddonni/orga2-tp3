
Los sistemas operativos son el software que se ocupa de manejar y administrar los recursos del hardware y proveer servicios a los programas. Sin sistemas operativos sólidos, realizar la mayoría de las tareas que realizamos hoy con las computadoras serían imposibles.

En este trabajo nos proponemos aprender y entender como funcionan los mecanismos básicos que implementa un sistema operativo para manejar la memoria, manejar interrupciones, alternar tareas, etc.

A lo largo de la historia existieron muchos paradigmas de sistemas operativos, en general atados a las capacidades tecnológicas de la época. 
En este TP nos proponemos hacer un sistema operativo que trabaja en modo protegido de 32 bits, con paginación y multitarea.
Teniendo esas características, nuestro sistema operativo \emph{de juguete} es más avanzado en varios aspectos que el DOS original de Microsoft (que trabajaba en modo real y era monotarea), por lo que sentimos que va a ser un interesante desafío.
\\

Para testear nuestro kernel durante el proceso de desarrollo utilizamos el software \texttt{Bochs}, un proyecto de código abierto, que permite emular una IBM PC tanto de 32 bits, como de 64 bits. Tambien permite emular dispositivos y un BIOS, por lo cual es ideal para el desarrollo de sistemas operativos.

