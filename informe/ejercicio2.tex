\subsubsection*{Inicialización de la IDT}

\par Para inicializar la IDT se llama a una funci\'on en lenguaje C, "void idt\_inicializar(void)". La IDT se representa mediante un arrelgo (de tama\~no 255) de "idt\_entry"
(esta estructura fue definida como lo muestra el extracto de C\'odigo \ref{idtEntry}, seg\'un las especificaciones dadas en el manual de Intel). Esta funci\'on utiliza un macro
(que se encuentra en el extracto de C\'odigo \ref{macro}) para inicializar cada una de las entradas necesarias de la IDT. 
\par El macro define el offset como los 16 bits menos y m\'as significativos (ya que \'este se encuentra separado en dos campos de 16 bits cada uno) de la direcci\'on de la tarea de atenci\'on
de la interrupci\'on correspondiente. El selector de segmento lo define como 0x0040, que es 0x8 (el indice del segmento de codigo de nivel de privilegio 0 en la GDT) shifteado 3 bits a la izquierda.
Los atributos los define como 0x8e00, que representan un segmento presente (P = 1), un nivel de privilegio de 0 (DPL = 00), un tama\~no de Gate de 32 bits (bits 8 a 12 de la Interrupt Gate = 0b01110),
y los 7 bits restantes en 0. 
\par Cabe destacar la entrada de la IDT correspondiente a la interrupci\'on 0x46 (correspondiente a las \textit{syscalls}), que se inicializa de la forma detallada en el extracto de c\'odigo \ref{0x46}
. La \'unica diferencia entre esta entrada y la definida por el macro previamente mencionado son los atributos, puntualmente el DPL (3 en esta entrada y 0 en las otras); esta diferencia se debe a que la \textit{syscall} debe poder ser llamada por las tareas, mientras que las otras interrupciones deben estar reservadas al nivel m\'as privilegiado.
\par La rutina de atenci\'on de cada interrupci\'on se genera a partir de un macro (que se encuentra en el extracto de C\'odigo 5 que imprime el c\'odigo de error correspondiente
a pantalla y luego queda en un loop infinito (esta rutina es un placeholder, que luego ser\'a reemplazada por una que mate a la tarea que causa la interrupci\'on y llame a la idle).
\par La IDT ya inicializada se puede acceder tras ejecutar la instrucci\'on $lidt [IDT\_DESC]$. El descriptor de la IDT, "IDT\_DESC" es una estructura (definida como se ve en el extracto de
c\'odigo 6) que contiene el tama\~no de la IDT y su direcci\'on en memoria.

\begin{lstlisting} [caption={Estructura de idt\_entry},label=idtEntry]
typedef struct str_idt_entry_fld {
    unsigned short offset_0_15;
    unsigned short segsel;
    unsigned short attr;
    unsigned short offset_16_31;
} __attribute__((__packed__, aligned (8))) idt_entry;

\end{lstlisting}


\begin{lstlisting} [caption={Codigo del macro utilizado para inicializar la IDT},label=macro]
#define IDT_ENTRY(numero)                                                                                        
    idt[numero].offset_0_15 = (unsigned short) ((unsigned int)(&_isr ## numero) & (unsigned int) 0xFFFF);        
    idt[numero].segsel = (unsigned short) 0x0040;                                                                
    idt[numero].attr = (unsigned short) 0x8e00;                                                                  
    idt[numero].offset_16_31 = (unsigned short) ((unsigned int)(&_isr ## numero) >> 16 & (unsigned int) 0xFFFF);
\end{lstlisting}

\begin{lstlisting} [caption={Codigo del macro utilizado para la rutina de atencion de interrupciones},label=macroISR]
_isr%1:
    mov eax, %1
    push dword 0x0f0f 
    push dword 0
    push dword 0
    push MENSAJE_ERROR_%1
    call print
    
    jmp \$
\end{lstlisting}

\begin{lstlisting} [caption={Estructura de IDT\_Desc}],label=idtDesc] 
typedef struct str_idt_descriptor {
    unsigned short  idt_length;
    unsigned int    idt_addr;
} __attribute__((__packed__)) idt_descriptor;
\end{lstlisting}

\begin{lstlisting} [caption={Codigo de la entrada de IDT de int 0x46},label=0x46]
#define IDT_ENTRY(numero)                                                                                        
    idt[numero].offset_0_15 = (unsigned short) ((unsigned int)(&_isr ## numero) & (unsigned int) 0xFFFF);        
    idt[numero].segsel = (unsigned short) 0x0040;                                                                
    idt[numero].attr = (unsigned short) 0xee00;                                                                  
    idt[numero].offset_16_31 = (unsigned short) ((unsigned int)(&_isr ## numero) >> 16 & (unsigned int) 0xFFFF);
\end{lstlisting}


