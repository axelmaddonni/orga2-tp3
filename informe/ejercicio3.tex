\subsubsection*{Inicialización de de directorio y tablas de kernel}

\par Para inicializar el directorio del kernel, lo que hacemos es, en la primera posición declarar la tabla de kernel (que será identity mapping), y luego ponemos todo el resto de directorio en 0 (bit de presente en 0, que indica que esas entradas no direccionan nada).

\par Luego inicializamos la tabla de kernel, que está en identity mapping, como dijimos anteriormente. Eso es bastante fácil, ya que podemos usar la misma variable para iterar y para decir la dirección física a la que direccionará una dirección virtual.

\subsubsection*{Activación de Paginación}

\par Para habilitar paginación, una vez inicializado correctamente el mmu, activamos el bit de PE del registro cr0:
\begin{lstlisting} [caption={Paginación en kernel.asm}],label=paginacion] 
    mov eax, cr0
    or eax, 0x80000000
    mov cr0, eax
\end{lstlisting}
    
\subsubsection*{Imprimir en pantalla}

\par Para pintar la pantalla usamos las funciones que nos permiten pintar rectángulos, que no necesitan explicación dado que son muy simples. (más de screen.h en funciones auxiliares)