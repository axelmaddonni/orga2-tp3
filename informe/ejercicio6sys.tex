\subsubsection*{Syscalls}
\par Los sistemas operativos proveen servicios (en la forma de \textit{system calls}, o \textit{syscalls}) para que sus tareas puedan realizar ciertas acciones que de otra forma no podr\'ian. En este caso, nuestro sistema operativo consta con una sola \textit{syscall}, mapeada a la interrupci\'on 0x46, que realizar\'a distintas acciones basado en el par\'ametro que se pase por el registro EAX.
Las tres acciones que la \textit{syscall} permite realizar a las tareas son:
\begin{itemize}
    \item Moverse (EAX = 0x1)
    \item Cavar (EAX = 0x2)
    \item Preguntar su posici\'on (EAX = 0x3)
\end{itemize}

\subsubsection*{Rutina de atenci\'on de interrupciones}
\par Las tareas deben realizar una interrupci\'on, int 0x46, para acceder a los servicios de la \textit{syscall}; en el registro EAX deben pasar el tipo de servicio requerido (0x1 para moverse, 0x2 para cavar y 0x3 para pedir su posici\'on), y en ECX el par\'ametro del servicio (mover y posici\'on toman un par\'ametro adicional).
La rutina de atenci\'on de la interrupci\'on (cuyo c\'odigo se puede ver en el extracto 15) 0x46 (70 en decimal) pushea los registros necesarios y llama a la funci\'on (implementada en lenguaje C) "game\_syscall\_manejar", y luego salta a la tarea idle mediante un jump far 0x70:0. Esta funci\'on llama a otras, que corresponden a cada servicio, con los par\'ametros necesarios; si el \textit{syscall} devuelve un error (resultado ~0, es decir la negaci\'on l\'ogica de 0x0, el mayor resultado posible en un entero sin signo), se ocupa de llamar a la funci\'on que lo resuelva.

\begin{lstlisting} [caption={Rutina de atenci\'on de la interrupci\'on 0x46},label=int]
_isr70:
  push ecx
  push edx
  push ebx
  push esp
  push ebp
  push esi
  push edi
  pushfd
  
  push ecx
  push eax
  call game_syscall_manejar
  add esp, 8

  jmp 0x70:0 ;voy a idle
 
  popfd
  pop edi
  pop esi
  pop ebp
  pop esp
  pop ebx
  pop edx
  pop ecx
  
  iret

\end{lstlisting}



\subsubsection*{Mover}
\par La funci\'on "game_syscall_pirata_mover" toma como par\'ametros la id del pirata que se quiere mover, y uno de cuatro valores posibles que representan la direcci\'on en la cual desea moverse. 
Convierte la id en un puntero al pirata mediante la funci\'on "id_pirata2pirata", y guarda su posici\'on; si el puntero que devuelve es nulo (ya que la id no corresponde a un pirata v\'alido), devuelve un error.
Si el pirata est\'a intentando moverse fuera del mapa, devuelve un error, en caso contrario actualiza su posici\'on.
Luego, genera dos arreglos, de tama\~no 3 cada uno, con el componente x e y de las posiciones de mapa que descubrir\'a al moverse (de forma que el producto cartesiano de los dos arreglos provee todas las posiciones que va a descubrir).
En caso de ser un explorador, se pintan (del color del jugador) las posiciones previamente inexploradas que fueron descubiertas en ese movimiento; adicionalmente, si se descubre un bot\'in con monedas, se pinta una 'o' del color del jugador que lo descubri\'o, y se lanza (o encola, si no hay lugar para otro pirata de ese jugador) un minero.
Luego, se mapean para cada pirata las p\'aginas de memoria correspondientes a las posiciones reci\'en exploradas (como se ve en el extracto de c\'odigo 16).
Adicionalmente, se pinta en pantalla la nueva posici\'on. Finalmente, se mapea la posici\'on fisica correspondiente a la nueva posici\'on del pirata a la direcci\'on virtual 0x400000, y se copia el c\'odigo de la tarea a \'esta (como se ve en el extracto de c\'odigo 17).



\begin{lstlisting} [caption={Mapeo de las posiciones de memoria reci\'en descubiertas},label=mapeo]


uint p;
for(p = 0; p<MAX_CANT_PIRATAS_VIVOS; p++)
{
    if(jug->vivos[p])
    {
        for(i = 0; i<3; i++)
        {
            for(h = 0; h<3; h++)
            {			
                if(en_rango(explorado_x[i], explorado_y[h]))
                {
                    uint ind = (explorado_y[h]*MAPA_ANCHO+explorado_x[i]) * 0x1000;
                    mmu_mapear_pagina(0x800000+ind, (pde *) jug->piratas[p].cr3, 0x500000+ind, 0, 1);
                }
            }
        }
    }
}

\end{lstlisting}


\begin{lstlisting} [caption={Mapeo de y copiado a la posici\'on actual},label=copiado]    

uint indice_viejo = (y_viejo*MAPA_ANCHO+x_viejo) * 0x1000;                                
uint indice_nuevo = (pir->posicion[1]*MAPA_ANCHO+pir->posicion[0]) * 0x1000;          
                                                                                          
mmu_mapear_pagina(0x400000, (pde *) pir->cr3, 0x500000+indice_nuevo, 1, 1);               
copiar_pagina(0x800000+indice_viejo, 0x400000);	

\end{lstlisting}

\subsubsection*{Cavar}
\par La funci\'on "game_syscall_cavar" recupera el puntero al pirata a partir de su id (y devuelve un error si el puntero es nulo o apunta a un explorador).
Luego recorre el arreglo de botines (como se ve en el extracto de c\'odigo 18) y, de encontrarse uno en la misma posici\'on que el minero, aumenta en uno el puntaje del jugador y decrementa, tambi\'en en uno, la cantidad de monedas en el bot\'in.
Si la cantidad de monedas en el bot\'in es menor o igual a 0, o si no hay ning\'un bot\'in en la posici\'on en la que esta intentando cavar, devuelve un error.
Devuelve un entero sin signo que representa la cantidad de monedas que quedan en el bot\'in.

\begin{lstlisting} [caption={Extracci\'on de monedas del bot\'in},label=cavar]

uint i;
for (i = 0; i < BOTINES_CANTIDAD; i++)
{
    if (x == botines[i][0] && y == botines[i][1])
    {
        if (botines[i][2] > 0) 
        {
            (pir->jugador)->monedas++;
            screen_pintar_puntajes();
            return --botines[i][2];
        }
        else return ~0;
    }
}

\end{lstlisting}


\subsubsection*{Posici\'on}
\par La funci\'on "game_syscall_pirata_posicion" devuelve la posici\'on del pirata que causa la interrupci\'on (si el \'indice de entrada es -1) o del pirata que se encuentre en la \'indice-i\'esima posici\'on del arreglo de piratas del jugador. Devuelve error si la id no pertenece a un pirata v\'alido (es decir si "id_pirata2pirata" devuelve null), si el \'indice no pertenece al rango [-1, 7], o si el \'indice no corresponde a un pirata vivo. La posici\'on la devuelve como un solo entero sin signo, resultado de la expresi\'on detallada en el extracto de c\'odigo 19.

\begin{lstlisting} [caption={Devolver la posici\'on},label=posicion]

return (y << 8 | x);

\end{lstlisting}
